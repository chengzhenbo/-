\documentclass[a4paper,12pt]{ctexart}
\input{report-sp24}

\newcommand{\studentName}{张三} % 学生姓名
\newcommand{\studentID}{20242100000} % 学生学号
\newcommand{\studentGrade}{计算机科学与技术 2024} % 学生班级

\newcommand{\reportName}{最小生成树算法比较}
\newcommand{\className}{算法分析与设计}
\newcommand{\reportType}{课程报告}
\newcommand{\reportSemester}{2024春}
\newcommand{\prof}{程振波}

\begin{document}
\maketitle
\newpage

\section{内容与要求}
本次课程报告以最小生成树问题为背景,请根据以下要求完成报告内容:
\begin{itemize}
    \item 问题背景请首先介绍什么是最小生成树问题,然后描述最小生成树问题的应用场景;
    \item 最小生成树算法介绍部分,请分别描述Prim算法和Kruskal算法,算法描述请使用伪代码。伪代码的格式可参考如下欧拉算法\ref{alg:euclid}的描述;
    \begin{algorithm}
        \caption{欧拉算法}\label{alg:euclid}
        \begin{algorithmic}[1]
        \Procedure{Euclid}{$a,b$}\Comment{a和b的公共}
        \State $r\gets a\bmod b$
        \While{$r\not=0$}\Comment{如果r是0}
        \State $a\gets b$
        \State $b\gets r$
        \State $r\gets a\bmod b$
        \EndWhile\label{euclidendwhile}
        \State \textbf{return} $b$\Comment{返回公共因子}
        \EndProcedure
        \end{algorithmic}
    \end{algorithm}
    \item 算法实现和比较部分包括:
    \begin{itemize}
        \item 要求分别使用不同数据结构实现Prim算法(数组和优先队列)和Kruskal算法(数组与并查集),并给出不同数据结构实现这两个算法运行时间的变化;
        \item 请给出选用不同数据结构对算法效率影响的理论分析;
        \item 比较Prim算法和Kruskal算法在求解稠密图和稀疏图时运行时间的比较;
        \item 算法比较需要随机生成至少10组不同的数据,然后统计每组不同数据的算法运行平均时间;
        \item 算法实现不要贴源代码,只需通过图表的方式给出不同情形(如数据结构)下的数据(如运行时间);
    \end{itemize}
    \item 总结部分需要根据算法实现与比较的数据分析得出一般性的结论;
    \item 课程报告要求语言精炼,数据翔实,列出参考文献;
    \item 课程报告要求独立完成。
\end{itemize}

该课程报告的模版是在CTex\cite{ctex}基础上修改而成,如何在模版插入图片或者制作表格请参考该模版的文档。

\section{问题背景}

\section{最小生成树算法介绍}
\subsection{Prim算法}

\subsection{Kruskal算法}

\section{算法实现与比较}
\subsection{算法性能提升}

\subsection{算法适用场景分析}

\section{结论}
\begin{thebibliography}{9}
    \bibitem{ctex} https://ctex.org/
\end{thebibliography}
\end{document}